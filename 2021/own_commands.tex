%=============================================================
%~~~~~~~~~~~~~~~~~~~~~~~~~~~~~~~~~~~~~~~~~~~~~~~~~~~~~~~~~~~~~
%=============================================================
%
% 						COMMANDS
%
%=============================================================
%~~~~~~~~~~~~~~~~~~~~~~~~~~~~~~~~~~~~~~~~~~~~~~~~~~~~~~~~~~~~~
%=============================================================


%=============================================================
% Renewing the chapter command
%=============================================================

\let\oldchapter\chapter
\renewcommand{\chapter}[1]{%
    \oldchapter{#1}%
%
    \thispagestyle{empty}%
}

%=============================================================
% Renewing the printindex command
%=============================================================

\let\oldprintindex\printindex
\renewcommand{\printindex}{%
    \renewcommand{\chapter}[1]{\oldchapter{##1}}
    
    \oldprintindex
    
    \renewcommand{\chapter}[1]{%
        \oldchapter{##1}%
%
        \thispagestyle{empty}%
    }
}

%=============================================================
% Song and chapter import commands
%=============================================================
\newcommand{\includechapter}[1]{\include{chapters/#1}}

\newcommand{\inputsong}[1]{\input{songs/#1}}

\newcommand{\inputnotes}[1]{\input{notes/#1}}

\newcommand{\inputextra}[1]{\input{extra/#1}}


%=============================================================
% Image placing commands
%=============================================================

\newcommand{\fullpic}[1]{\vspace*{\fill}\begin{center}\includegraphics[width=\linewidth]{#1}\end{center}\vspace*{\fill}}%


% Command used to remove the 'pt' part of point values
\makeatletter\def\strippt{\strip@pt}\makeatother
\newsavebox{\Image}

\newlength{\ImageH}
\newlength{\ImageW}
\newlength{\dx}
\newlength{\dy}

\newcommand{\pic}[4]{%
%
% #1 = dx
% #2 = dy
% #3 = image width (for includegraphics)
% #4 = image path
%
%	ex:		\pic{10pt}{-10mm}{50cm}{test.png}
%
% Reading the image
	\savebox{\Image}{\includegraphics[width=#3]{#4}}
% Getting the width and height of the image
	\settowidth{\ImageW}{\usebox{\Image}}
	\settoheight{\ImageH}{\usebox{\Image}}
% Converting the deltas to 'pt' units
	\setlength{\dx}{#1}
	\setlength{\dy}{#2}	
% Setting the unitlenght to 1pt, to only use 'pt' units
	\setlength{\unitlength}{1pt}
	\noindent
% When using the different values, strip them of the 'pt' part, to only use the value
	\begin{picture}(0, 0)%(\strippt\ImageW, \strippt\ImageH)
		\put(\strippt\dx, \strippt\dy){\usebox{\Image}}
	\end{picture}
% The picture takes a bit of vertical space, so we adjust it a litte bit back
	\vspace{-.77cm} % the value is carefully measured to be just correct, DON'T CHANGE IT!!
}


%=============================================================
% Special headings (Helan, Halvan, Tersen, etc.) commands
%=============================================================
\newcommand{\heading}[1]{\begin{LARGE}\scshape{\textbf{#1}}\end{LARGE}\vspace{.5cm}}



%=============================================================
% Song author details
%=============================================================

% Redefine the cite commands, to be without the italic font
\DeclareFieldFormat*{citetitle}{#1}
\DeclareFieldFormat*{citeauthor}{#1}

\newcommand{\songtext}[1]{%
		\begin{footnotesize}
			\textit{Text:} #1
		\end{footnotesize}
	}%
\newcommand{\original}[1]{%
		\begin{footnotesize}
			\textit{Melodi:} #1
		\end{footnotesize}
	}%
\newcommand{\musik}[1]{%
		\begin{footnotesize}
			\textit{Musik:} #1
		\end{footnotesize}
	}%
	
	
% Command used by the old Note-sheets
\newcommand{\kompos}[1]{%
		\textit{#1}
	}%
	


%=============================================================
% Verse counter command
%	(should get zeroed every time a song is created)
%=============================================================

\newcounter{versenumber}
\setcounter{versenumber}{1}
\newcommand{\showversenumber}{%
	\makebox[0pt][r]{\arabic{versenumber}.\hskip 5pt}%
	\addtocounter{versenumber}{1}%
}%
	
	
	
%=============================================================
% Remove "Kapitel X" from upper corner
%=============================================================

\renewcommand{\chaptermark}[1]{\markboth{\scshape{#1}}{}}


%=============================================================
% For adding a phrase to the index. Make sure the phrase fits
% 	on one line.
%=============================================================
\newcommand{\addsongnametoindex}[1]{\index{#1}}
\newcommand{\addphrasetoindex}[1]{\index{\emph{#1}}}


%=============================================================
% Start a new double page, with text on the left.
%=============================================================
\newcommand*\newdoublepage{%
  \newpage
  \ifodd\value{page}\hbox{}\newpage\fi
}


%=============================================================
%~~~~~~~~~~~~~~~~~~~~~~~~~~~~~~~~~~~~~~~~~~~~~~~~~~~~~~~~~~~~~
%=============================================================
%
% 						ENVIRONMENTS
%
%=============================================================
%~~~~~~~~~~~~~~~~~~~~~~~~~~~~~~~~~~~~~~~~~~~~~~~~~~~~~~~~~~~~~
%=============================================================



%=============================================================
% Used to indicate repeat of a line of text in the song
%=============================================================
\newcommand{\repetitionbegin}{/:}
\newcommand{\repetitionend}{:/}
\ExplSyntaxOn
\NewDocumentEnvironment{repetition}{o}%
{%
	\repetitionbegin{}
}%
{%
	\repetitionend{}\IfValueT{#1}{\,x#1}
}%
\ExplSyntaxOff



%=============================================================
% Used to create a new Song
%	Arguments
%	-m	Song Title
%=============================================================
\ExplSyntaxOn
\NewDocumentEnvironment{song}{m o}%
{%
% Sets the verse counter to 1 whenever the song environment is used.
\setcounter{versenumber}{1}
% Adds the title to the table of contents
	\IfValueTF{#2}{%
		\phantomsection\label{#2}
	}{%
		\phantomsection\label{#1}\addcontentsline{toc}{section}{#1}\index{#1}
	}
\begin{large}
	\textbf{\scshape{#1}}
\end{large}
}%
{\vspace{.8cm}}%
\ExplSyntaxOff




%=============================================================
% Environment for aligning tables containing song
% instructions
%=============================================================
\ExplSyntaxOn
\NewDocumentEnvironment{alternatinglyrics}{o}%
{%
    \hspace{-.2cm}
    \IfValueTF{#1}{%
        \begin{tabular}{l l l}
    }{%
        \begin{tabular}{l l}
    }%
}%
{%
\end{tabular}%
}%
\ExplSyntaxOff

