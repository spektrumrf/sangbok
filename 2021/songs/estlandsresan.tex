\begin{song}{Estlandsresan}
	
	\original{\citetitle{metsavendade_laul}}\\
	\songtext{\citeauthor{mccann}}
	
	\addphrasetoindex{Ack, jag har knappast råd med brännvin}
	\addphrasetoindex{Ai-tših-ai-tšah}
	
	\; \emph{ Berättande:}\\
	Ack, jag har knappast råd med brännvin\\
	så jag till söderlandet for;\\
	i Estland har jag råd med brännvin;\\
	där mötte jag mitt livs amour.
	
	\vspace{-.1cm}
	\; \emph{ Passionerat:}	\\
	Ai-tših-ai-tšah, ai-tšah, kom bröder,\\
	kom systrar, sjung med oss en sång:\\
	Ai-tših-ai-tšah, ai-tšah, nu glöder\\
	vår kärleksanda än en gång!
	
	\vspace{.1cm}
	\; \emph{ Glatt:}\\
	Se, fastän jag för spriten brinna,\\
	blev något annat mer centralt:\\
	Jag mötte där en estnisk kvinna,\\
	Vi älskade, ja helt brutalt!

	\vspace{-.1cm}
	\; \emph{ Melankoliskt:}\\
	Ai-tših-ai-tšah, ai-tšah, hon var den\\
	vildaste mö jag nånsin mött,\\
	Ai-tših-ai-tšah, ai-tšah, men jag minns än:\\
	vi sedan skiljdes stup i ett.

	\vspace{.1cm}
	\; \emph{ Beklagande:}\\
	Och hennes svarta vinbärsögon,\\
	jag blir nog aldrig minnet kvitt,\\
	för hennes svarta vinbärsögon,\\
	ja dom har stulit hjärtat mitt.
	
	\vspace{-.1cm}
	\; \emph{ Bedrövat:}\\
	Ai-tših-ai-tšah, ai-tšah, den sorgen\\
	är nog min största än idag.\\
	Ai-tših-ai-tšah, ai-tšah, bröstkorgen\\
	hennes får hugna fler än jag.

	\vspace{.1cm}
	\; \emph{ Tillfrisknat, upplivat:}\\
	Ett lyckligt slut får ändå sagan,\\
	och dess final en fin slutkläm.\\
	Nej, ta och hejda er bekalagan,\\
	jag tog ju brännvin med mig hem!

	\vspace{-.1cm}
	\; \emph{ Ystert, stående:}\\
	Ai-tših-ai-tšah, ai-tšah, så res er!\\
	Kom bröder, häll i systrars glas,\\
	Ai-tših-ai-tšah, ai-tšah, ja res er!\\
	drick bägar’n ur, kom i extas!
	
\end{song}
