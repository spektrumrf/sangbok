\begin{song}{Fredmans epistel nr 82}
	
%	\songtext{C M Bellman}
	
	\songtext{\citeauthor{bellman}}
	
	\addphrasetoindex{Hvila vid denna källa}
	\addphrasetoindex{Vila vid denna källa}
	
	\showversenumber
	Hvila vid denna källa,\\
	vår lilla Frukost vi framställa:\\
	rödt Vin med Pimpinella\\
	och en nyss skuten Beccasin.\\
	Klang hvad Buteljer, Ulla!\\
	I våra Korgar öfverstfulla,\\
	tömda i gräset rulla,\\
	och känn hvad ångan dunstar fin,\\
	ditt middags Vin\\
	sku vi ur krusen hälla,\\
	med glättig min.\\
	Hvila vid denna källa,\\
	hör våra Valdthorns klang Cousine.\\
	\emph{Corno\ldots{} (con bocca chiusa)}\\
	Valdthornens klang Cousine.
	
	\showversenumber
	Prägtigt på fältet pråla,\\
	än Hingsten med sitt Sto och Fåla,\\
	än Tjurn han höres vråla,\\
	och stundom Lammet bräka tör;\\
	Tuppen på taket hoppar,\\
	och liksom Hönan vingen loppar,\\
	svalan sitt hufvud doppar,\\
	och Skatan skrattar på sin stör.\\
	Lyft Kitteln; hör.\\
	Lät Caffe-glöden kola,\\
	där nedanför.\\
	Prägtigt på fältet pråla\\
	de ämnen som mest ögat rör.\\
	\emph{Corno\ldots{}}\\
	Som mest vårt öga rör.
	
	\showversenumber
	Himmel! hvad denna Runden,\\
	af friska Löfträn sammanbunden,\\
	vidgar en plan i Lunden,\\
	med strödda gångar och behag.\\
	Ljufligt där löfven susa,\\
	i svarta hvirflar grå och ljusa,\\
	träden en skugga krusa,\\
	inunder skyars fläkt och drag.\\
	Tag, Ulla tag,\\
	vid denna måltids stunden,\\
	ditt glas som jag.\\
	Himmel! hvad denna Runden,\\
	bepryds af blommor tusen slag!\\
	\emph{Corno\ldots{}}\\
	Af blommor tusen slag.
	
	\showversenumber
	Nymphen, se hvar hon klifver,\\
	och så beställsam i sin ifver,\\
	än Ägg och än Oliver,\\
	uppå en rosig tallrik bär.\\
	Stundom en sked hon öser,\\
	och öfver Bunken gräddan slöser;\\
	Floret i barmen pöser,\\
	då hon den Mandeltårtan skär.\\
	En Kyckling där,\\
	af den hon vingen rifver,\\
	nyss kallnad är.\\
	Nymphen se hvar hon klifver,\\
	och svettas i et kärt besvär.\\
	\emph{Corno\ldots{}}\\
	Och svettas i besvär.
	
	\pic{5.7cm}{-1.3cm}{.52\textwidth}{bellman.png}
	
	\showversenumber
	Blåsen J Musikanter,\\
	vid Eols blåst från berg och branter;\\
	Sjungen små Kärleks-Panter,\\
	bland gamla Mostrars kält och gnag.\\
	Syskon! en sup vid disken,\\
	och pro secundo en på Fisken;\\
	Krögarn, den Basilisken,\\
	summerar Taflan full i dag.\\
	Klang Du och Jag!\\
	Klang Ullas amaranther,\\
	af alla slag!\\
	Blåsen J Musikanter,\\
	och hvar och en sin kallsup tag.\\
	\emph{Corno\ldots{}}\\
	Hvar en sin kallsup tag.
	
	\showversenumber
	Ändtlig i detta gröna,\\
	får du mitt sista afsked röna;\\
	Ulla! farväl min Sköna,\\
	vid alla Instrumenters ljud.\\
	Fredman ser i minuten\\
	sig til Naturens skuld förbruten,\\
	Clotho ren ur Surtouten,\\
	afklipt en knapp vid Charons bud.\\
	Kom hjertats Gud!\\
	At Fröjas ätt belöna\\
	med Bacchi skrud.\\
	Ändtlig i detta gröna,\\
	stod Ulla sista gången Brud.\\
	\emph{Corno\ldots{}}\\
	Den sista gången Brud.
	
\end{song}
