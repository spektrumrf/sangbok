\begin{song}{O, gamla klang- och jubeltid}
	
%	\songtext{August Lind}\\
%	\musik{Eugen Höfling}
	
	\songtext{\citeauthor{lind}}\\
	\musik{\citeauthor{hofling}}
	
	\showversenumber
	O, gamla klang- och jubeltid,\\
	ditt minne skall förbliva,\\
	och än åt livets bistra strid\\
	ett rosigt skimmer giva.\\
	Snart tystnar allt vårt ystra skämt,\\
	vår sång blir stum, vårt glam förstämt.\\
	\begin{repetition}
		O, jerum, jerum, jerum,\\
		o, quae mutatio rerum!
	\end{repetition}
	
	\showversenumber
	Var äro de som kunde allt\\
	blott ej sin ära svika,\\
	\begin{alternatinglyrics}[3]
		&	\emph{Män}:			& som voro män av äkta halt,\\
		&	\emph{Alla andra}:	& och världens herrar lika?\\
	\end{alternatinglyrics}\\
	De drogos bort från vin och sång\\
	till vardagslivets tråk och tvång.\\
	\begin{repetition}
		O, jerum, jerum, jerum,\\
		o, quae mutatio rerum!
	\end{repetition}
	
	
    \begin{alternatinglyrics}[3]
        \showversenumber	& \emph{Fysiker}:		& Den ene ser partikelhord\\
							&						& bland vektorfält vibrera.\\
							& \emph{Dataloger}:		& Den andre sliter tangentbord\\
							&						& och bitar transporterar.\\
							& \emph{Kemister}:		& En kokar sånt som luktar phult,\\
							& \emph{Matematiker}:	& en skriver $\varphi$ och $\chi$ förstult.
    \end{alternatinglyrics}\\
	\begin{repetition}
		O, jerum, jerum, jerum,\\
		o, quae mutatio rerum!
	\end{repetition}
	
	\showversenumber
	Men hjärtat i en sann student\\
	kan ingen tid förfrysa,\\
	den glädjeeld, som där har tänt,\\
	hans hela liv skall lysa.\\
	Det gamla skalet brustit har,\\
	men \emph{kärnan} finnes frisk dock kvar,\\
	\begin{repetition}
		och vad han än må mista,\\
		den skall dock aldrig brista!
	\end{repetition}
	
	\showversenumber
	Så sluten, bröder, fast vår krets\\
	till glädjens värn och ära!\\
	Trots allt vi tryggt och väl tillfreds\\
	vår vänskap trohet svära.\\
	Lyft bägarn högt och klinga vän!\\
	De \emph{gamla gudar} leva än\\
	\begin{repetition}
		bland skålar och pokaler!
	\end{repetition}
	
\end{song}
