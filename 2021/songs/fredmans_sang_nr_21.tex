\begin{song}{Fredmans sång nr 21}
	
%	\songtext{C M Bellman}
	
	\songtext{\citeauthor{bellman}}
	
	\addphrasetoindex{Så lunka vi så småningom}

	% GAMMAL SVENSKA
	
	\showversenumber
	Så lunka vi så småningom\\
	från Bacchi buller och tumult,\\
	när döden ropar, Granne kom,\\
	ditt timglas är nu fullt.\\
	Du Gubbe fäll din krycka ner,\\
	och du, du Yngling, lyd min lag,\\
	den skönsta Nymph som åt dig ler\\
	inunder armen tag.
	
	Tycker du at grafven är för djup,\\
	nå välan så tag dig då en sup,\\
	tag dig sen dito en, dito två, dito tre,\\
	så dör du nöjdare.
	
	\showversenumber
	Du vid din remmare och präss,\\
	rödbrusig och med hatt på sned,\\
	snart skrider fram din likprocess\\
	i några svarta led;\\
	Och du som pratar där så stort,\\
	med band och stjernor på din rock,\\
	ren snickarn kistan färdig gjort,\\
	och hyflar på des lock.\\
	Tycker du at \ldots{}
	
	\showversenumber
	Men du som med en trumpen min,\\
	bland riglar, galler, järn och lås,\\
	dig hvilar på ditt penningskrin,\\
	innom din stängda bås;\\
	Och du som svartsjuk slår i kras\\
	buteljer, speglar och pocal;\\
	Bjud nu god natt, drick ut dit glas,\\
	Och helsa din rival;\\
	Tycker du at \ldots{}
	
	\showversenumber
	Och du som under titlars klang\\
	din tiggarstaf förgylt hvart år,\\
	som knappast har, med all din rang,\\
	en skilling til din bår;\\
	Och du som ilsken, feg och lat,\\
	fördömmer vaggan som dig hvälft,\\
	och ändå dagligt är placat\\
	til glasets sista hälft;\\
	Tycker du at \ldots{}
	
	\showversenumber	
	Du som vid Martis fältbasun\\
	i blodig skjorta sträckt ditt steg;\\
	Och du som tumlar i paulun,\\
	i Chloris armar feg;\\
	Och du som med din gyldne bok\\
	vid templets genljud reser dig,\\
	som rister hufvud lärd och klok,\\
	och för mot afgrund krig;\\
	Tycker du at \ldots{}
	
	\showversenumber
	Men du som med en ärlig min\\
	plär dina vänner häda jämt,\\
	och dem förtalar vid dit vin,\\
	och det liksom på skämt;\\
	Och du som ej försvarar dem,\\
	fastän ur deras flaskor du,\\
	du väl kan slicka dina fem,\\
	hvad svarar du väl nu?\\
	Tycker du at \ldots{}
	
	\showversenumber
	Men du som til din återfärd,\\
	ifrån det du til bordet gick,\\
	ej klingat för din raska värd,\\
	fastän han ropar: Drick!\\
	Drif sådan gäst från mat och vin,\\
	kör honom med sitt anhang ut,\\
	och sen med en ovänlig min,\\
	ryck remmarn ur hans trut.\\
	Tycker du at \ldots{}
	
	\showversenumber
	Säg är du nöjd? min granne säg,\\
	så prisa värden nu til slut;\\
	Om vi ha en och samma väg,\\
	så följoms åt; drick ut.\\
	Men först med vinet rödt och hvitt\\
	för vår Värdinna bugom oss,\\
	och halkom sen i grafven fritt,\\
	vid aftonstjernans bloss.\\
	Tycker du at \ldots{}


%	% NY SVENSKA
%	
%	\showversenumber
%	Så lunka vi så småningom\\
%	från Bacchi buller och tumult,\\
%	när döden ropar, Granne kom,\\
%	ditt timglas nu är fullt!\\
%	Du, gubbe, fäll din krycka ner,\\
%	och du, du yngling, lyd min lag,\\
%	den skönsta nymf som åt dig ler,\\
%	inunder armen tag!
%	
%	Tycker du att graven är för djup,\\
%	nå välan så tag dig då en sup,\\
%	tag dig sen dito en, dito två, dito tre,\\
%	så dör du nöjdare.
%	
%	\showversenumber
%	Du vid din remmare och präss,\\
%	rödbrusig och med hatt på sned,\\
%	snart skrider fram din likprocess\\
%	i några svarta led;\\
%	Och du som pratar där så stort,\\
%	med band och stjärnor på din rock,\\
%	ren snickarn kistan färdig gjort,\\
%	och hyvlar på dess lock.\\
%	Tycker du att \ldots{}
%	
%	\showversenumber
%	Men du som med en trumpen min,\\
%	bland riglar, galler, järn och lås,\\
%	dig vilar på ditt penningskrin,\\
%	inom din stängda bås;\\
%	Och du som svartsjuk slår i kras\\
%	buteljer, speglar och pokal;\\
%	Bjud nu godnatt, drick ur ditt glas,\\
%	och hälsa din rival;\\
%	Tycker du att \ldots{}
%	
%	\showversenumber
%	Och du som under titlars klang\\
%	din tiggarstav förgylt vart år,\\
%	som knappast har, med all din rang,\\
%	en skilling till din bår;\\
%	Och du som ilsken, feg och lat,\\
%	fördömmer vaggan som dig välvt,\\
%	och ändå dagligt är plakat\\
%	till glasets sista hälft;\\
%	Tycker du att \ldots{}
%	
%	\showversenumber
%	Du som vid Martis fältbasun\\
%	i blodig skjorta sträckt ditt steg;\\
%	Och du som tumlar i paulun,\\
%	i Chloris armar feg;\\
%	Och du som med din gyllne bok\\
%	vid templets genljud reser dig,\\
%	som rister huvud lärd och klok,\\
%	och för mot avgrund krig;\\
%	Tycker du att \ldots{}
%	
%	\showversenumber
%	Men du som med en ärlig min\\
%	plär dina vänner häda jämt,\\
%	och dem förtalar vid ditt vin,\\
%	och det liksom på skämt;\\
%	Och du som ej försvarar dem,\\
%	fastän ur deras flaskor du,\\
%	du väl kan slicka dina fem,\\
%	vad svarar du väl nu?\\
%	Tycker du att \ldots{}
%	
%	\showversenumber
%	Men du som till din återfärd,\\
%	ifrån det du till bordet gick,\\
%	ej klingat för din raska värd,\\
%	fastän han ropar: Drick!\\
%	Driv sådan gäst från mat och vin,\\
%	kör honom med sitt anhang ut,\\
%	och sen med en ovänlig min,\\
%	ryck remmarn ur hans trut.\\
%	Tycker du att \ldots{}
%	
%	\showversenumber
%	Säg är du nöjd? min granne säg,\\
%	så prisa värden nu till slut;\\
%	Om vi har en och samma väg,\\
%	så följoms åt; drick ut.\\
%	Men först med vinet rött och vitt,\\
%	för vår värdinna bugom oss,\\
%	och halkom sen i graven fritt,\\
%	vid aftonstjärnans bloss.\\
%	Tycker du att \ldots{}
	
\end{song}
