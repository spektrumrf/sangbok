\begin{song}{Sjösala vals}
	
	\vspace{-.1cm}
	
	\addphrasetoindex{Rönnerdahl han skuttar med ett skratt ur sin säng}

    \showversenumber	
	Rönnerdahl han skuttar med ett skratt ur sin säng.\\
	Solen står på Orrberget. Sunnanvind brusar.\\
	Rönnerdahl han valsar över Sjösala äng.\\
	Hör min vackra visa, kom sjung min refräng!\\
	Tärnan har fått ungar och dyker i min vik.\\
	Ur alla gröna dungar hörs finkarnas musik,\\
	och se, så många blommor som redan slagit ut på ängen!\\
	Gullviva, mandelblom, kattfot och blå viol.
	
	\vspace{-.05cm}
	
    \showversenumber
	Rönnerdahl han virvlar sina lurviga ben\\
	under vita skjortan som viftar kring vaderna.\\
	Lycklig som en lärka uti majsolens sken,\\
	sjunger han för ekorrn, som gungar på gren!\\
	-- Kurre, kurre, kurre! Nu dansar Rönnerdahl!\\
	Koko! Och göken ropar uti hans gröna dal\\
	och se, så många blommor som redan slagit ut på ängen!\\
	Gullviva, mandelblom, kattfot och blå viol.
	
	\vspace{-.05cm}
	
    \showversenumber
	Rönnerdahl han binder utav blommor en krans,\\
	binder den kring håret, det gråa och rufsiga,\\
	valsar in i stugan och har lutan till hands,\\
	väcker frun och barnen med drill och kadans.\\
	-- Titta! ropar ungarna, Pappa är en brud,\\
	med blomsterkrans i håret och nattskjorta till skrud!\\
	Och se, så många blommor som redan slagit ut på ängen!\\
	Gullviva, mandelblom, kattfot och blå viol.
	
	\vspace{-.05cm}
	
    \showversenumber
	Rönnerdahl är gammal, men han valsar ändå!\\
	Rönnerdahl har sorger och ont om sekiner.\\
	Sällan får han rasta -- han får slita för två.\\
	Hur han klarar skivan, kan ingen förstå --\\
	ingen, utom tärnan in viken (hon som dök)\\
	och ekorren och finken och vårens första gök\\
	och blommorna, de blommor som redan slagit ut på ängen.\\
	Gullviva, mandelblom, kattfot och blå viol.
	
\end{song}
