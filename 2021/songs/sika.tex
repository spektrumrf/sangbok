\begin{song}{Sika}
	
%	\songtext{Juice Leskinen}\\
%	\musik{Juice Leskinen}
	
	\songtext{\citeauthor{leskinen}}\\
	\musik{\citeauthor{leskinen}}

    \showversenumber	
	Joulun viettomme alkaa jo toukokuussa,\\
	porsaalla silloin on herkut suussa.\\
	Se lihoo, se kasvaa, se vahvistuu,\\
	ja sitten kun koittaa marraskuu,\\
	niin lapset laulaa: ''Joulun tähti on ehdoton.''\\
	Sika - se kuulan kalloon saa,\\
	Sika - sen setä teurastaa.\\
	Sika - ja setä verta juo,\\
	Sika - se Joulumielen tuo.
	
    \showversenumber
	Minä läävässä lojuvaa karjua katsoin\\
	kärsivin ilmein, vellovin vatsoin.\\
	Jouluna sikaa mä muista en,\\
	vain kinkkua vahtaan himoiten.\\
	ja lapset laulaa: ''Joulun tähti on ehdoton.''\\
	Sika - kun sitä suolataan,\\
	Sika - yhdessä kuolataan.\\
	Sika - ja isi mielellään,\\
	Sika - on vesi kielellään.
	
    \showversenumber
	Jouluruuhkassa ihminen sokkona ryntää\\
	kadulla adventtisohjoa kyntää.\\
	Joku ihmistä katsoo, kääntää pään,\\
	kun ihminen rypee läävässään.\\
	ja lapset laulaa: ''Joulun tähti on ehdoton.''\\
	Sika - se kohta paistetaan,\\
	Sika - sen perää maistetaan.\\
	Sika - on sähköuunissa,\\
	Sika - on mutsi duunissa.
	
    \showversenumber
	Kato, äiti on laittanut kystäkyllä,\\
	sinappihuntu on sialla yllä,\\
	isoveli veitsensä terottaa,\\
	ja luistansa porsaan erottaa.\\
	Lapset laulaa: ''Joulun tähti on ehdoton.''\\
	Sika - lopulta tappoi sen,\\
	Sika - ympäristö vatsan happoisen.\\
	Sika - voi kuinka isi syö,\\
	Sika - isiltä loppuu vyö.
	
    \showversenumber
	Joulu täynnä on kinkkua, kylkeä, potkaa,\\
	läskiä, niskaa ja tietenkin votkaa,\\
	kun loppuu aatto ja alkaa yö,\\
	niin ihminen on sitä mitä hän syö.\\
	Sika - la la la laa la laa,\\
	Sika - tra la la laa la laa.\\
	Sika - lala la laa la laa,\\
	Sika - tra la la laa la laa.
	
\end{song}
