\begin{song}{SI-enheter}
	
	\original{\citetitle{studentsangen}}
	
	\addphrasetoindex{lumen, Couloumb, meter, Newtonsekund}
	
	\si{\lumen} \si{\coulomb} \si{\weber} \si{\newton\second}\\
	\si{\candela} \si{\tesla} \si{\pascal} \si{\volt}/\si{\sievert}\\
	\si{\watt}/\si{\katal} \si{\meter} \si{\lux} \si{\decibel}\\
	\si{\steradian}/\si{\becquerel}\\
	\si{\joule}/\si{\kelvin} \si{\ohm} \si{\watt}/\si{\degreeCelsius}\\
	\si{\kilogram}/\si{\radian} \si{\gray} \si{\hertz}\\
	\si{\henry} \si{\ampere} \si{\electronvolt}/\si{\siemens}\\
	\si{\meter}/\si{\square\second} \si{\meter}/\si{\square\second} \si{\farad}

	\emph{Fusklapp}:\\
	Lumen Coulomb Weber Newtonsekund\\
	Candela Tesla Pascal Volt per Sievert\\
	Watt per Katal meter Lux decibel\\
	steradianer per Becquerel\\
	Joule per Kelvin Ohm, Watt per grad Celsius\\
	kilogram per radianer Gray Hertz\\
	Henry Ampere elektronvolt per Siemens\\
	\repetitionbegin{} Meter per sekund i kvadrat\repetitionend{} Farad!
	
\end{song}
