\begin{song}{Oh alte Burschenherrlichkeit}

	\original{\citetitle{o_gamla_klang_och_jubeltid}}

    \showversenumber	
	Oh alte Burschenherrlichkeit,\\
	wohin bist du entschwunden?\\
	Nie kehrst du wieder goldne Zeit,\\
	so froh und ungebunden!\\
	Vergebens spähe ich umher,\\
	ich finde deine Spur nicht mehr.\\
	\begin{repetition}
		O jerum, jerum, jerum\\
		o, quae mutatio rerum.
	\end{repetition}
	
    \showversenumber
	Den Burschenhut bedeckt der Staub,\\
	es sank der Flaus in Trümmer.\\
	Der Schläger ward des Rostes Raub,\\
	verblichen ist sein Schimmer.\\
	Verklungen der Kommersgesang,\\
	verhallt Rapier- und Sporenklang.\\
	\begin{repetition}
		O jerum, jerum, jerum\\
		o, quae mutatio rerum.
	\end{repetition}
	
    \showversenumber
	Wo sind sie, die vom breiten Stein,\\
	nicht wankten und nicht wichen,\\
	die ohne Moos bei Scherz und Wein,\\
	den Herrn der Erde glichen?\\
	Sie zogen mit gesenktem Blick,\\
	in das Philisterland zurück.\\
	\begin{repetition}
		O jerum, jerum, jerum\\
		o, quae mutatio rerum.
	\end{repetition}
	
    \showversenumber
	Da schreibt mit finsterm Amtsgesicht,\\
	der eine Relationen.\\
	Der andere seufzt beim Unterricht,\\
	und der macht Rezensionen;\\
	Der schilt die sünd'ge Seele aus,\\
	und der flickt ihr verfallnes Haus.\\
	\begin{repetition}
		O jerum, jerum, jerum\\
		o, quae mutatio rerum.
	\end{repetition}
	
    \showversenumber
	Allein das rechte Burschenherz,\\
	kann nimmermehr erkalten;\\
	Im Ernste wird, wie hier im Scherz,\\
	der rechte Sinn stets walten;\\
	Die alte Schale nur ist fern,\\
	geblieben ist uns doch der Kern,\\
	\begin{repetition}
		und den lasst fest uns halten.
	\end{repetition}
	
    \showversenumber
	Drum Freunde reichet euch die Hand,\\
	damit es sich erneure,\\
	der alten Freundschaft heil'ges Band,\\
	das alte Band der Treue.\\
	Klingt an und hebt die Gläser hoch,\\
	die alten Burschen leben noch,\\
	\begin{repetition}
		noch lebt die alte Treue.
	\end{repetition}
	
\end{song}
