\begin{song}{Kemistvisan}
	
%	\songtext{Guss Mattsson}\\
%	\musik{Guss Mattsson}
	
	\songtext{\citeauthor{mattsson}}\\
	\musik{\citeauthor{mattsson}}

    \showversenumber	
	Bland tekniska ynglingar alla\\
	kemisten är kronan ändå\\
	ty en gång, då världarna falla,\\
	han tittar blott glättigt därpå.\\
	\begin{repetition}
		Ty destillation\\
		och sublimation\\
		ger ragnarök, kropparnas konstitution
	\end{repetition}
	
    \showversenumber
	Jag varit i olika länder,\\
	jag sett båd' Paris och Berlin.\\
	Jag skålat vid Rhenflodens stränder\\
	i Pommerys välkända vin.\\
	\begin{repetition}
		Möblerat jag har\\
		två lyckliga dar\\
		hos turken, som aldrig på kärleken spar
	\end{repetition}
	
    \showversenumber
	Nu bliver det tyst i salen\\
	blott flaskor och Burken står kvar.\\
	Nu tystna de glättiga talen\\
	och tömt blivit fröjdernas kar.\\
	\begin{repetition}
		I källaren blott\\
		sker än lite klott\\
		ty det är ju alltid en källares lott
	\end{repetition}
	
    \showversenumber
	Och sen, när vi alla är döda,\\
	trikalciumortofosfat,\\
	och slutat vår jordiska möda,\\
	och tömt blivit fröjdernas fat.\\
	\begin{repetition}
		Dock själen så fin\\
		har kvar energin\\
		tack vare kinetiska gasteorin
	\end{repetition}
	
\end{song}
